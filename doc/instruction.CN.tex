\documentclass[10p,UTF8]{ctexart}
\usepackage{amsmath}
\usepackage{setspace}
\usepackage{caption}
%\usepackage{subfigure}
\usepackage{graphicx, subfig}
\usepackage{float}
\usepackage{indentfirst}
\usepackage{listings}
\usepackage{xcolor}
\usepackage{enumerate}
\usepackage[colorlinks,linkcolor=blue]{hyperref}

\lstset{
    columns=fixed,       
    numbers=left,                                        % 在左侧显示行号
    frame=none,                                          % 不显示背景边框
    backgroundcolor=\color[RGB]{245,245,244},            % 设定背景颜色
    keywordstyle=\color[RGB]{40,40,255},                 % 设定关键字颜色
    numberstyle=\footnotesize\color{darkgray},           % 设定行号格式
    commentstyle=\it\color[RGB]{0,96,96},                % 设置代码注释的格式
    stringstyle=\rmfamily\slshape\color[RGB]{128,0,0},   % 设置字符串格式
    showstringspaces=false,                              % 不显示字符串中的空格
    language=bash,                                        % 设置语言
}

\title{JDSurfG 使用说明}
\author{Nanqiao Du}
\date{\today}

\begin{document}
   \maketitle
   \tableofcontents
   \newpage
   \section{Introduction}
   JDSurfG 包含了三个模块,分别用来计算球坐标系下的重力矩阵,直接面波反演和面波-重力联合反演。
   其中重力部分基于自适应Gauss-Legendre数值积分方法,面波部分基于面波一步法计算面波走时敏感核。
   该方法通过岩石物理中的经验关系式构建S波速度和密度的转换公式,从而将重力数据和面波走时数据
   统一到一个反演框架中。该方法能够用来反演地壳-上地幔内的S波速度结构。

   该程序由c++和Fortran混编写成,用c++来构建反演框架,封装数据,处理I/O,并计算重力矩阵和
   面波一维敏感核,Fortran主要负责计算面波走时和面波二维敏感核。

   \section{Preliminaries}
   该程序使用\href{http://eigen.tuxfamily.org/index.php?title=Main_Page}{Eigen}来进行多维
   数组的存取,并用\href{http://www.gnu.org/software/gsl/}{GSL}库来计算Gauss-Legendre节点
   的位置和权重。因此务必在您的计算机上装好这两个库。同时,请确保您的g++编译器
   支持c++11标准(gcc $\geq$4.8.1)。

    \section{Installation}
    下载.zip包之后,按如下方式解压
    \begin{lstlisting}[language=bash]
        unzip JDSurfG.zip
    \end{lstlisting}
    如果您没有将Eigen和GSL的include (或) lib文件夹添加到环境变量中,那么请在JDSurfG/include/Makefile中添加
    他们的路径,例如:
    \begin{lstlisting}[language=c]
        path_to_eigen= -I/path-to-eigen
    \end{lstlisting}
    之后,通过如下方式编译
    \begin{lstlisting}[language=bash]
        cd JDSurfG
        make
    \end{lstlisting}
    之后,你会惊人地发现在bin文件夹下,会出现三个可执行文件
    \begin{lstlisting}[language=bash]
       mkmat DSurfTomo JointSG  
    \end{lstlisting}

   \section{面波直接反演模块}
   该模块需要三个输入文件(如果需要做检测版,则需要四个):
   \begin{itemize}
       \item DSurfTomo.in: 包含模型信息、使用的频散的信息,阻尼等
       \item surfdataSC.dat : 面波频散数据
       \item MOD: 初始模型
       \item MOD.true: 检测板模型
   \end{itemize}
   在说明输入格式之前,有几点注意的地方:
   \begin{enumerate}[(1)]
        \item 该模型在经纬度方向是均匀网格,在深度方向的网格间距可以变化
       \item 输入的模型要比反演区域大,具体来说,如果需要反演的区域在区域:
              $(lat_0 \sim lat_1,lon_0\sim lon_1,0\sim dep_1)$之间,则输入模型的区域是:
              $(lat_0-dlat \sim lat_1+dlat, lon_0-dlon \sim lon_1+dlon,0 \sim dep_1-dz)$
        \item 台站不要放在模型边界区域。不然有可能射线追踪时部分射线跑到模型边界上,引起计算误差。
   \end{enumerate}
   下面分别说明几个文件的输入格式:
   \subsection{参数文件}
   例子中的参数文件为 DSurfTomo.in,它是一个self-explantory的文件,
   \begin{itemize}
       \item 第1行 三个数依次是\textbf{纬度方向节点个数},\textbf{经度方向节点个数}
                   以及\textbf{深度方向节点个数}
       \item 第2行 左上角位置的经纬度(注意这是\textbf{输入模型}的左上角!)
       \item 第3行 纬度、经度网格间距(度)
       \item 第4行 反演的正则化系数和阻尼
       \item 第5行 因为输入模型是网格节点,需要将其转换为层状介质计算频散,该参数表示将节点变为层状介质需要在节点间插入几层,
                    通常来讲取为3足够。
       \item 第6行 模型的最大和最小速度
       \item 第7行 最大迭代次数
       \item 第8行 使用的全部Rayleigh波相速度周期个数,如果没使用则置零。
       \item 第9行 如果8不为0,将所有周期列出来,否则依次进行Rayleigh群速度,Love相速度、群速度
       \item 第?行 是否进行检测板测试?如果是填1
       \item ?  noiselevel, 加入噪声的大小(注意此处是和数据的相对大小):
       \[   
           d_i = d_{syn} + (d_{syn} * noiselevel * N(0,1))
        \]
   \end{itemize}

   \subsection{初始模型和检测板模型文件}
   例子中初始模型和检测板模型文件分别为MOD和MOD.true, 这两个文件格式稍有不同。主要差别在第一行上。\\
   MOD: 第一行存放所有z方向节点的位置(km)。MOD.true不用放。
   之后的行存放输入的速度模型:每一行,存放一个给定深度的,每个给定经度的,所有纬度上的速度(km/s)。
   即速度按行优先方法存为 V(nz,nlon,nlat)。第i行存放 V(:,:,i);

   \subsection{频散数据文件}
   例子中频散数据文件为surfdataSC.dat,用于存放面波频散数据(台站间距/面波走时)。 
   如下我们展示了该文件的前九行:\\
    \#  31.962600 108.646000 1 2 0\\
    33.732800 105.764100 3.0840\\
    34.342500 106.020600 3.1154\\
    33.357400 104.991700 3.0484\\
    33.356800 106.139500 3.0820\\
    34.128300 107.817000 3.1250\\
    33.229300 106.800200 3.0575\\
    \#  29.905000 107.232500 1 2 0\\
    34.020000 102.060100 2.7532\\

   可以看出,该文件包括面波激发台站(带\#号)和接收台站(不带\#号)。
   对于激发台站,该行的格式为 \# lat lon period-index wavetype velotype:
   \begin{itemize}
       \item wavetype: 表示面波类型, Rayleigh波为2,Love 为1
       \item velotype: 表示速度类型, 相速度为0,群速度为1
       \item period-index: 表示该面波群/相速度的周期的序号,比如在Rayleigh相速度中用到的
                            周期为 4s,5s,6s,7s,而该台站对之间的数据的周期是5s,
                            则倒数三个数分别为2,2,0。
   \end{itemize}
   每一个激发台站后面紧跟,在该面波类型,该周期下,所有有频散数据的接收台站,每一行依次为 lat lon v。

   \subsection{运行方法}
   将上述文件准备到一个文件夹中,进入该文件夹,命令行执行:
   \begin{lstlisting}[language=bash]
    ./DSurfTomo -h
   \end{lstlisting}
   获取帮助,然后按提示输入即可。

   \subsection{输出文件}\label{面波输出文件}
   该程序共有两个输出文件夹: kernel/和results/,在kernel中保存了面波伪3-D敏感核文件,results中
   保存了每一次迭代后的模型文件和面波走时文件。下面逐次介绍两个文件夹。
   \begin{itemize}
    \item 文件夹kernel: 保存了面波伪3-D敏感核。其中txt文件个数=并行使用的线程数(见\ref{线程数} 节),编号从0开始。
           每一个txt文件的格式是:数据编号,模型参数编号(只包括要反演的部分)以及敏感核的大小。
    \item 文件夹results:保存了每一次迭代的模型文件(mod\_iter\*)以及面波走时文件(res\*.dat)。
            其中模型文件的格式为:经度、纬度、深度、S波速度。面波走时文件的格式为:台站间距,观测
            到时、理论到时。
    \end{itemize}

   \section{联合反演模块}
   该模块需要4-6个输入文件:
    \begin{itemize}
        \item JointSG.in: 包含模型信息、使用的频散的信息,阻尼,联合反演权重等。
        \item surfdataSC.dat : 面波频散数据。
        \item obsgrav.dat : 重力异常数据。
        \item gravmat.dat : 重力异常对密度的敏感核矩阵。
        \item MOD: 初始模型。
        \item MOD.true: 检测板模型,用于检测板测试。
        \item MOD.ref : 速度参考模型,用于计算重力异常理论值。
    \end{itemize}
    下面逐次介绍这几个文件。

    \subsection{参数文件}
    JointSG.in和DSurfTomo.in的文件格式基本相同。不同之处在于最后两行。为了理解其参数的意义,
    这里列出了联合反演的misfit function:
    \[
        L = \frac{p}{N_1 \sigma_1^2} (d_1- d_1^o)^2 + 
            \frac{1-p}{N_2 \sigma_2^2} (d_2- d_2^o)^2 \tag{1}
    \]
    \begin{itemize}
        \item 倒数第二行: 参数p, 介于0-1之间,p=0表示重力单独反演。由于位场反演不唯一性较高,
                        p值通常取得比0.5大。
        \item 倒数第一行 : 面波和重力的均方差。
    \end{itemize}

    \subsection{重力数据文件}
    重力数据文件比较简单,将所有重力数据,按照每行分别为经度,纬度和该点的重力异常排成一个文件
    即可。
    下面给出了该文件前9行:\\
    100.000000 35.000000 -224.206921\\
    100.000000 34.950000 -224.110699\\
    100.000000 34.900000 -241.774731\\
    100.000000 34.850000 -240.245968\\
    100.000000 34.800000 -234.187542\\
    100.000000 34.750000 -246.670605\\
    100.000000 34.700000 -238.575939\\
    100.000000 34.650000 -241.357250\\
    100.000000 34.600000 -260.059429\\

    \subsection{重力矩阵文件}
    该文件通过重力模块生成(见第6节)

    \subsection{速度参考模型}
    由于使用的重力异常是重力绝对大小相对正常椭球体的重力异常。因此在实际计算中,需要归算到局部异常
    中。因此,在程序中对观测值进行了去均值处理。每次迭代时,计算理论重力异常的方法分三步:
    \begin{itemize}
        \item 计算当前模型与参考模型分别对应的密度值。
        \item 计算密度异常,之后用重力密度矩阵获取该异常对应的重力值。
        \item 对所获得的重力异常去均值。
    \end{itemize}
    该模型可以用第三方密度模型代替,也可以用程序默认的处理(对初始模型的每个深度求平均值)。

    \subsection{运行方法}
    将上述文件准备到一个文件夹中,进入该文件夹,命令行执行:
    \begin{lstlisting}[language=bash]
     ./JointSG -h
    \end{lstlisting}
    获取帮助,然后按提示输入即可。

    \subsection{输出文件}
    该程序共有两个输出文件夹: kernel/和results/。其中kernel的格式与\ref{面波输出文件} 节一致。\\
    文件夹results:保存了每一次迭代的模型文件(joint\_mod\_iter\*),面波走时文件(res\_surf\*.dat),和
    重力异常文件(res\_grav*.dat)。面波走时文件和模型文件和 \ref{面波输出文件} 节一致,重力异常文件的格式
    为:重力异常观测值,重力异常理论值。
 

    \section{重力模块}
    该模块需要如下几个输入文件:
    \begin{itemize}
        \item JointSG.in 或 DSurfTomo.in : 包含模型网格信息
        \item MOD(和MOD.true) : 初始速度模型和真实速度模型
        \item obsgrav.dat : 重力数据观测值和相应的经纬度
    \end{itemize}
    上述格式在之前已经说明过,在此不再赘述。

    \subsection{运行方法}
    将上述文件准备到一个文件夹中,进入该文件夹,命令行执行:
    \begin{lstlisting}[language=bash]
     ./mkmat -h
    \end{lstlisting}
    获取帮助,然后按提示输入即可。
    \subsection{输出文件}
    该程序共有一个输出文件:gravmat.dat。为重力异常的密度敏感核。格式为:数据编号、模型参数编号和
    敏感核大小。


    \section{一些小脚本}

   \section{高级选项}
    \subsection{并行使用的线程数}\label{线程数}
    在include/openmp.hpp中有一行:
    \begin{lstlisting}[language=c++]
    #define nthread 4
    \end{lstlisting}
    将4改为其他线程数即可。

    \subsection{速度-密度经验关系的选择}
    在src/utils/empirical.f90中 
    \begin{lstlisting}[language=fortran]
    subroutine empirical_relation(vsz,vpz,&
        rhoz)bind(c,name="empirical_relation")
        use,intrinsic :: iso_c_binding
      
        real(c_float), intent(in) ::  vsz
        real(c_float),intent(out) ::vpz,rhoz
      
        vpz=0.9409 + 2.0947*vsz - 0.8206*vsz**2+ &
                0.2683*vsz**3 - 0.0251*vsz**4
        rhoz=1.6612*vpz- 0.4721*vpz**2 + &
                0.0671*vpz**3 - 0.0043*vpz**4 + & 
                0.000106*vpz**5
      
    end subroutine empirical_relation
    \end{lstlisting}
    将经验关系换为其他即可。如果要进行联合反演,注意要修改该文件中相应的导数函数。

    \subsection{Gauss-Legendre节点个数的选择以及重力矩阵的计算范围}
    在src/gravity/gravmat.cpp中,可以修改最大和最小节点个数,重力矩阵不为0的最大距离
    (超过该范围的两个点之间的吸引力可以忽略不记)
    \begin{lstlisting}[language=c++]
    #define NMIN 5
    #define NMAX 256
    #define maxdis 100.0 
    \end{lstlisting}
    
    \subsection{面波走时正演计算的网格间距}
    由于使用的网格不一定能准确计算面波走时,因此在计算过程中,需要
    用一套新的网格来进行走时正演计算。该计算网格是通过将模型网格进一步划分来实现的,即通过
    在模型网格中间插入一些新的点。这些新的点连同模型网格构成一个新的计算网格。插入的点数可以在
    src/SurfaceWave/sphfmm.f90 的synthetic和CalFrechet函数中修改(参数gdx和gdy,默认为8)。
    纬度和经度方向分别插入的点数为gdx+1和gdy+1,插入这些点后,计算网格将在
    纬度方向和经度方向分别有(nlat-1) * gdx +1 和 (nlon+1) * gdy + 1个节点,

\end{document}